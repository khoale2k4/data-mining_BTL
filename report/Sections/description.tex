\section{Mô tả bài toán}

\subsection{Bối cảnh và vấn đề}

Trong ngành bán lẻ, việc dự báo doanh số bán hàng là một trong những bài toán quan trọng và thách thức nhất. 
Đối với một tập đoàn quy mô toàn cầu như \textbf{Walmart}, việc dự báo chính xác có ảnh hưởng trực tiếp đến hiệu quả hoạt động kinh doanh.

\paragraph{Bối cảnh:}
Hoạt động của một chuỗi siêu thị bán lẻ phụ thuộc vào nhiều yếu tố biến động liên tục như:
các ngày lễ trong năm (ví dụ: Giáng sinh, Lễ Tạ ơn), 
các chương trình khuyến mãi (\textit{MarkDown}), 
và các yếu tố kinh tế -- xã hội bên ngoài (như giá nhiên liệu, tỷ lệ thất nghiệp).

\paragraph{Vấn đề:}
\begin{itemize}
    \item \textbf{Tối ưu tồn kho:} 
    Nếu dự báo quá cao, Walmart sẽ bị tồn đọng hàng hóa, tăng chi phí lưu kho (đặc biệt với hàng hóa dễ hỏng). 
    Ngược lại, nếu dự báo quá thấp, cửa hàng sẽ bị \textit{cháy hàng} (\textit{out-of-stock}), 
    dẫn đến mất doanh thu và giảm sự hài lòng của khách hàng.
    
    \item \textbf{Quản lý nhân sự:} 
    Dự báo doanh số giúp ban quản lý cửa hàng sắp xếp lịch làm việc và số lượng nhân viên phù hợp với lượng khách hàng dự kiến, 
    tránh lãng phí chi phí nhân công hoặc thiếu người phục vụ.
\end{itemize}

Vì vậy, vấn đề đặt ra là cần xây dựng một mô hình dựa trên dữ liệu lịch sử để dự đoán doanh số bán hàng hàng tuần một cách chính xác, 
giúp Walmart đưa ra các quyết định kinh doanh hiệu quả hơn.

\subsection{Mô tả dữ liệu}

\paragraph{Nguồn dữ liệu:}
Bộ dữ liệu được sử dụng trong bài tập này có tên \textit{``Walmart Sales Forecast''}, 
được thu thập và công bố công khai trên nền tảng Kaggle 
(tại địa chỉ: \url{https://www.kaggle.com/datasets/aslanahmedov/walmart-sales-forecast}).

\paragraph{Mô tả dữ liệu:}
Bộ dữ liệu bao gồm thông tin bán hàng lịch sử của \textbf{45 cửa hàng Walmart} trong giai đoạn từ năm 2010 đến 2012. 
Dữ liệu được chia thành ba tệp chính như sau:

\begin{table}[H]
\centering
\caption{Mô tả tệp \texttt{stores.csv}}
\begin{tabular}{|l|l|p{8cm}|}
\hline
\textbf{Tên thuộc tính} & \textbf{Kiểu dữ liệu} & \textbf{Ý nghĩa} \\ \hline
Store & Số nguyên & Mã định danh của cửa hàng (từ 1 đến 45) \\ \hline
Type & Phân loại & Loại cửa hàng (A, B hoặc C) \\ \hline
Size & Số nguyên & Diện tích của cửa hàng \\ \hline
\end{tabular}
\end{table}

\begin{table}[H]
\centering
\caption{Mô tả tệp \texttt{features.csv}}
\begin{tabular}{|l|l|p{8cm}|}
\hline
\textbf{Tên thuộc tính} & \textbf{Kiểu dữ liệu} & \textbf{Ý nghĩa} \\ \hline
Store & Số nguyên & Mã cửa hàng (khóa ngoại liên kết với \texttt{stores.csv}) \\ \hline
Date & Ngày tháng & Ngày (theo tuần) \\ \hline
Temperature & Số thực & Nhiệt độ trung bình tại khu vực cửa hàng \\ \hline
Fuel\_Price & Số thực & Giá nhiên liệu trung bình tại khu vực \\ \hline
MarkDown1–5 & Số thực & Dữ liệu ẩn danh về 5 loại chương trình giảm giá (\textit{MarkDown}). Giá trị \texttt{NaN} nghĩa là không có giảm giá. \\ \hline
CPI & Số thực & Chỉ số giá tiêu dùng \\ \hline
Unemployment & Số thực & Tỷ lệ thất nghiệp tại khu vực \\ \hline
IsHoliday & Boolean & Đánh dấu tuần đó có phải tuần lễ đặc biệt hay không \\ \hline
\end{tabular}
\end{table}

\begin{table}[H]
\centering
\caption{Mô tả tệp \texttt{train.csv}}
\begin{tabular}{|l|l|p{8cm}|}
\hline
\textbf{Tên thuộc tính} & \textbf{Kiểu dữ liệu} & \textbf{Ý nghĩa} \\ \hline
Store & Số nguyên & Mã cửa hàng \\ \hline
Dept & Số nguyên & Mã định danh phòng ban (ví dụ: quần áo, điện tử, \ldots) \\ \hline
Date & Ngày tháng & Ngày (theo tuần) \\ \hline
Weekly\_Sales & Số thực & \textbf{Thuộc tính mục tiêu} (\textit{Target Variable}) – Doanh số bán hàng trong tuần \\ \hline
IsHoliday & Boolean & Đánh dấu tuần lễ đặc biệt \\ \hline
\end{tabular}
\end{table}

\subsection{Mục tiêu khai phá dữ liệu}

Dựa trên bối cảnh và bộ dữ liệu được cung cấp, các mục tiêu khai phá dữ liệu của nhóm được xác định như sau:

\paragraph{Mục tiêu chính:}
Xây dựng một mô hình \textbf{Hồi quy (Regression)} có khả năng dự đoán chính xác giá trị \texttt{Weekly\_Sales} (doanh số hàng tuần) cho từng phòng ban tại từng cửa hàng.

\paragraph{Mục tiêu phụ:}
\begin{itemize}
    \item Thực hiện phân tích dữ liệu khám phá (EDA) để hiểu rõ đặc điểm của dữ liệu và mối quan hệ giữa các biến 
    (ví dụ: doanh số tăng hay giảm vào ngày lễ? Nhiệt độ ảnh hưởng đến doanh số ra sao?).
    
    \item Phân tích và xác định các yếu tố (\textit{features}) có ảnh hưởng quan trọng nhất đến doanh số bán hàng 
    (ví dụ: \texttt{IsHoliday}, \texttt{MarkDowns}, hay \texttt{Size} của cửa hàng).
    
    \item So sánh và đánh giá hiệu quả của ít nhất \textbf{hai thuật toán Data Mining khác nhau} 
    để tìm ra mô hình có kết quả dự đoán tốt nhất.
\end{itemize}
